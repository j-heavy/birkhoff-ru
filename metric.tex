\chapter{Preliminaries}\label{chap:metr}
\addtocontents{toc}{\protect\begin{quote}}

\section*{Что такое аксиоматический подход?}
\addtocontents{toc}{What is the axiomatic approach?}

В аксиоматическом подходе плоскость определяется как всё, что удовлетворяет определённому списку свойств.
Эти свойства называются {}\emph{аксиомами}.
Аксиоматическая система для теории
всё равно что правила для игры.
Как только мы определились с системой аксиом, утверждение считается истинным, если оно следует из аксиом, и ничто другое таковым считаться не может.

Первые аксиомы были не совсем строгими.
Например, Евклид поисывал {}\emph{линию} как {}\emph{длину без ширины},
а {}\emph{прямую линию} как линию, {}\emph{все точки которой ровно лежат на себе}.
С другой стороны,
эти формулировки были достаточно ясны,
чтобы один математик мог понять другого.

Лучший способ понять аксиоматическую систему
— создать её самостоятельно.
Оглянитесь вокруг и выберите физическую модель
Евклидовой плоскости;
представьте себе бесконечную и совершенную поверхность меловой доски.
Теперь попробуйте собрать ключевые наблюдения
об этой модели.
Предположим, теперь у нас есть интуитивное понимает {}\emph{отрезка} и {}\emph{точки}.
\begin{enumerate}[(i)]
 \item\label{preaxiomI} Мы можем измерять расстояния между точками
 \item\label{preaxiomII} Мы можем провести только один отрезок,
проходящий через две заданные точки.
 \item\label{preaxiomIII} Мы можем измерять углы.
 \item\label{preaxiomIV} Если мы будем поворачиваться/сдвигаться, то не увидим разницу.
 \item\label{preaxiomV} Если мы изменим масштаб, то не заметим разницы.
\end{enumerate}
Этих наблюдений достаточно для начала.
Дальше мы будем развивать свой язык,
переформулируя их строго.

\section*{Что такое модель?}
\addtocontents{toc}{What is a model?}
\label{page:model}

Евклидова плоскость может быть строго определена следующим образом:

{}\emph{Определим {}\emph{точку} в Евклидовом пространстве как пару действительных чисел $(x,y)$,а {}\emph{расстояние} между двумя точками $(x_1,y_1)$ и $(x_2,y_2)$ будем вычислять по формуле:}
\[\sqrt{(x_1-x_2)^2+(y_1-y_2)^2}.\]

Вот так!
У нас есть {}\emph{численная модель} Евклидова пространства,
определённая на множестве действительных чисел.

Краткость является главным преимуществом модельного подхода,
но интуитивно не ясно, почему мы так определяем точки и расстояния между ними.

С другой стороны, наблюдения, сделанные в предыдущем разделе, интуитивно очевидны ---
это главное преимущество аксиоматического подхода.

Ещё одно преимущество заключается в том, что аксиоматический подход предполагает гибкость.
Например, мы можем убрать одну из аксиом
или поменять её на другую.
Мы будем проводить подобные манипуляции в главе~\ref{chap:non-euclid} и в дальнейшем.

\section*{Метрическое пространство}
\addtocontents{toc}{Metric spaces.}

Понятие метрического пространства позволяет нам
строго сказать {}\emph{``мы можем измерить дистанцию между двумя точками''}.
То есть, вместо (\ref{preaxiomI}) на странице \pageref{preaxiomI},
мы можем сказать: {}\emph{``Евклидово пространство является метрическим''}.

\begin{thm}{Definition}\label{def:metric-space}
Пусть $\mathcal X$ — непустое множество и 
$d$ — функция,
которая возвращает действительное число $d(A,B)$
для любой пары $A,B\in\mathcal X$.
Тогда $d$
мы назовём \index{metric}\emph{метркой} над 
$\mathcal X$, если для любых
$A,B,C\in \mathcal X$, выполняются следующие условия:
\begin{enumerate}[(a)]
\item\label{def:metric-space:a} Неотрицательность: 
$$d(A,B)\ge 0.$$
\item\label{def:metric-space:b} $A=B$ тогда и только тогда
$$d(A,B)=0.$$
\item\label{def:metric-space:c} Симметрия: $$d(A, B) = d(B, A).$$
\item\label{def:metric-space:d} Неравенство треугольника: 
$$d(A, C) \le d(A, B) + d(B, C).$$
\end{enumerate}
\index{metric space}\emph{Метрическое пространство} — непустое множество с определённой на нём метрикой. 
Более формально, матрическое пространство — это пара $(\mathcal X, d)$, $\mathcal X$ — множество и $d$ — метрика над ~$\mathcal X$.

Элементы $\mathcal X$ называются \index{point}\emph{точки} метрического пространства.
Для двух точек $A,B\z\in \mathcal X$, 
значение $d(A, B)$ будет называться \index{distance}\emph{расстояние} от $A$ до~$B$.
\end{thm}

\section*{Примеры}
\addtocontents{toc}{Examples.}

\begin{itemize}
\item {}\emph{Дискретная метрика.} Пусть $\mathcal X$ — произвольное множество. 
Для любых $A,B\z\in\mathcal X$, 
$d(A,B)\z=0$, если $A=B$ и $d(A,B)=1$ в остальных случаях.
Метрика $d$ называется \index{discrete metric}\emph{дискретной метрикой} над~$\mathcal X$.
\item\index{real line}\emph{Вещественная прямая.}
Множество всех действительных чисел ($\mathbb{R}$) с метрикой $d$ определённой как
$$d(A,B)\df|A-B|.$$
\end{itemize}

\begin{thm}{Упражнение}\label{ex:dist-square}
Докажите, что $d(A,B)=|A-B|^2$ {}\emph{не} является метрикой над $\mathbb{R}$.
\end{thm}

\begin{itemize}
\item {}\emph{Метрика на плоскости.}
Пусть $\mathbb{R}^2$ обозначает множество всех пар $(x,y)$ действительных чисел.
Допустим, $A=(x_A,y_A)$ и $B=(x_B,y_B)$.
Рассмотрим следующие метрики над $\mathbb{R}^2$:
\begin{itemize}
\item\index{Евклидова метрика{Euclidean metric}\emph{Евклидова метрика,} обозначаемая как \index{59@$d_1$, $d_2$, $d_\infty$}$d_2$ и определяемая как \label{def:d_2}
$$d_2(A,B)=\sqrt{(x_A-x_B)^2+(y_A-y_B)^2}.$$
\item\label{Manhattan plane}\index{Manhattan plane}\emph{Манхэттенское расстояние,} обозначаемое как $d_1$ и определяемое как
$$d_1(A,B)=|x_A-x_B|+|y_A-y_B|.$$
\item{}\emph{Расстояние Чебышёва,} обозначаемое как $d_\infty$ и определяемое как 
$$d_\infty(A,B)=\max\{|x_A-x_B|,|y_A-y_B|\}.$$
\end{itemize}
\end{itemize}

\begin{thm}{Упражнение}\label{ex:d_1+d_2+d_infty}
Докажите, что следующие функции являются метрикой над $\mathbb{R}^2$:
(a)~$d_1$; (b)~$d_2$; (c)~$d_\infty$.
\end{thm}


\section*{Сокращения для расстояния}
\addtocontents{toc}{Shortcut for distance.}

Большую часть времени,
мы изучаем только одну метрику в пространстве,
поэтому нам не нужно будет каждый раз называть метрику.

В метрическом пространстве $\mathcal X$,
расстояние между $A$ и $B$ в дальнейшем будет обозначаться как 
$$AB
\quad
\text{или}
\quad
d_{\mathcal X}(A,B);$$
последнее используется только в том случае, если нам нужно подчеркнуть, что $A$ и $B$ являются точками метрического пространства~$\mathcal X$.

Например, неравенство треугольника может быть записано как
$$AC\le AB+BC.$$

Для умножения мы будем использовать нотацию ``$\cdot$'',
в таком случае, мы не сможем спутать $AB$ с $A\cdot B$.

\section*{Изометрия, движение и отрезки}
\addtocontents{toc}{Isometries, motions, and lines.}

В этой части мы определим {}\emph{отрезки} в метрическом пространстве.
Как только мы это сделаем, утверждение {}\emph{``Мы можем провести только один отрезок, проходящий через две заданные точки.''} станет строгим; см. (\ref{preaxiomII}) на странице \pageref{preaxiomII}. 

Вспомним, что отоброжения $f\:\mathcal{X}\to\mathcal{Y}$
является \index{bijection}\emph{биективным},
если между двумя множествами можно установить взаимно однозначное соответствие.
Аналогично, $f\:\mathcal{X}\to\mathcal{Y}$ является биекцией, если существует \index{inverse}\emph{обратное};
отоброжение $g\:\mathcal{Y}\to\mathcal{X}$,
такое, что
$g(f(A))\z=A$ для любых $A\in\mathcal{X}$
и
$f(g(B))\z=B$ для любых $B\in\mathcal{Y}$. 

\begin{thm}{Определения}\label{def:isom}
Пусть $\mathcal X$ и $\mathcal Y$ — метрические пространства и $d_{\mathcal X}$, $d_{\mathcal Y}$ их метрики. 
Отоброжение
$$f\:\mathcal X \z\to \mathcal Y$$ 
называется \index{distance-preserving map}\emph{сохраняющим расстояние}, если 
$$d_{\mathcal Y}(f(A), f(B))
 = d_{\mathcal X}(A,B)$$
для любых $A,B\in {\mathcal X}$.

Биективное отоброжение, сохраняющие расстояние, называется \index{isometry}\emph{изометрией}. 

Два метрических пространства называются
\emph{изометричными}, если между ними существует изометрией из одного пространства в другое.

Изометрия из метрического пространства в себя же
называется \index{motion}\emph{движением} на плоскости.
\end{thm}

\begin{thm}{Упражнение}\label{ex:dist-preserv=>injective}
Покажите, что любое отоброжение, сохраняющие расстояние \index{injective map}\emph{инъективно};
то есть, если $f\:\mathcal X\to\mathcal Y$ является отоброженеим, сохраняющим расстояни, 
тогда $f(A)\ne f(B)$ для любых празличных пар точек $A, B\in \mathcal X$.
\end{thm}

\begin{thm}{Exercise}\label{ex:motion-of-R}
Show that if $f\:\mathbb{R}\to\mathbb{R}$ is a motion of the real line,
then either (a)
$f(x)=f(0)+x$ for any $x\in \mathbb{R}$, 
or (b)
$f(x)=f(0)-x$ for any $x\in \mathbb{R}$. 

\end{thm}

\begin{thm}{Exercise}\label{ex:d_1=d_infty}
Prove that $(\mathbb{R}^2,d_1)$ is isometric to $(\mathbb{R}^2,d_\infty)$.
\end{thm}

\begin{thm}{Advanced exercise}\label{ad-ex:motions of Manhattan plane}
Describe all the motions of the Manhattan plane, defined on page~\pageref{Manhattan plane}.
\end{thm}

If $\mathcal X$ is a metric space and $\mathcal Y$ is a subset of $\mathcal X$,
then a metric on $\mathcal Y$ can be obtained by restricting the metric from~$\mathcal X$. 
In other words, 
the distance between two points of $\mathcal Y$ is defined to be the distance between these points in $\mathcal X$.
This way any subset of a metric space can be also considered as a metric space. 

\begin{thm}{Definition}\label{def:line}
A subset $\ell$ of metric space is called a \index{line}\emph{line}, if it is isometric to the real line.
\end{thm}

A triple of points that lie on one line is called \index{collinear points}\emph{collinear}.
Note that if $A$, $B$, and  $C$ are  collinear, $AC\ge AB$ and $AC\ge BC$, then $AC= AB+BC$.

Some metric spaces have no lines, for example discrete metrics.
The picture shows examples of lines on the Manhattan plane $(\mathbb{R}^2,d_1)$. 
\begin{figure}[h!]
\centering
\includegraphics{mppics/pic-2}
\end{figure}

\begin{thm}{Exercise}\label{ex:y=|x|}
Consider the graph $y=|x|$ in $\mathbb{R}^2$.
In which of the following spaces 
(a) $(\mathbb{R}^2,d_1)$, 
(b) $(\mathbb{R}^2,d_2)$, 
(c) $(\mathbb{R}^2,d_\infty)$ 
does it form a line? 
Why?
\end{thm}

\section*{Half-lines and segments}
\addtocontents{toc}{Half-lines and segments.}

Assume there is a line $\ell$ passing thru
two distinct points $P$ and $Q$.
In this case we might denote $\ell$ as $(PQ)$.
There might be more than one line thru $P$ and $Q$,
but if we write \index{60@$(PQ)$, $[PQ)$, $[PQ]$}$(PQ)$ we assume that we made a choice of such line. 

\begin{thm}{Exercise}\label{ex:2mid}
How many points $M$ are there on the line $(A B)$ for which we have
\begin{enumerate}[(a)]
\item $AM= MB$ ?
\item $AM= 2\cdot MB$ ?
\end{enumerate}
\end{thm}

Let $[P Q)$ denotes the \index{half-line}\emph{half-line}
that starts at $P$ and contains~$Q$. 
Formally speaking, $[P Q)$ is a subset of $(P Q)$ which corresponds to $[0,\infty)$ under an isometry $f\:(P Q)\to \mathbb{R}$ such that $f(P)=0$ and $f(Q)>0$.

The subset of line $(P Q)$ between $P$ and $Q$ is called the \index{segment}\emph{segment} between $P$ and $Q$ and denoted by~$[P Q]$.
Formally, the segment can be defined as the intersection of two half-lines: $[P Q]=[P Q)\cap[Q P)$.

\begin{thm}{Exercise}\label{ex:trig==}
Show that 
\begin{enumerate}[(a)]
\item if $X\in [PQ)$, then 
$QX=|PX-PQ|$;
\item if $X\in [PQ]$, then 
$QX+XQ=PQ$.
\end{enumerate}

\end{thm}


\section*{Angles}
\addtocontents{toc}{Angles.}

Our next goal is to introduce {}\emph{angles} and {}\emph{angle measures}; 
after that, the statement {}\emph{``we can measure angles''} will become rigorous;
see (\ref{preaxiomIII}) on page~\pageref{preaxiomIII}.

An ordered pair of half-lines that start at the same point is called an \index{angle}\emph{angle}.
The angle $AOB$ (also denoted by \index{10@$\angle$}$\angle AOB$) is the pair of half-lines $[OA)$ and $[OB)$.
In this case the point $O$ is called the \index{vertex of the angle}\emph{vertex} of the angle.

Intuitively, the angle measure tells how much one has to rotate the first half-line counterclockwise, so it gets the position of the second half-line of the angle. 
The full turn is assumed to be $2\cdot\pi$;
it corresponds to the angle measure in radians.%
\footnote{For a while you may think that $\pi$ is a positive real number that measures the size of a half turn in certain units. Its concrete value $\pi\approx 3.14$ will not be important for a long time.}

The angle measure of $\angle AOB$ is denoted by \index{12@$\measuredangle$}$\measuredangle AOB$;
it is a real number in the interval $(-\pi,\pi]$. 

\begin{wrapfigure}{o}{25mm}
\vskip-0mm
\centering
\includegraphics{mppics/pic-4}
\end{wrapfigure}

The notations $\angle AOB$ and $\measuredangle AOB$ look similar;
they also have close but different meanings which better not be confused.
For example, the equality 
$\angle AOB=\angle A'O'B'$
means that
$[OA)=[O'A')$ and $[OB)\z=[O'B')$;
in particular, $O=O'$.
On the other hand the equality 
$\measuredangle AOB\z=\measuredangle A'O'B'$ 
means only equality of two real numbers;
in this case $O$ may be distinct from~$O'$.

Here is the first property of angle measure which will become a part of the axiom.

\textit{Given a half-line $[O A)$ and $\alpha\in(-\pi,\pi]$ there is a unique half-line $[O B)$ such that $\measuredangle A O B= \alpha$.}





\section*{Reals modulo $\bm{2\cdot\pi}$}
\addtocontents{toc}{Reals modulo $2\cdot\pi$.}



Consider three half-lines starting from the same point, $[O A)$, $[O B)$, and $[O C)$.
They make three angles $A O B$, $B O C$, and $A O C$,
so the value $\measuredangle A O C$ should coincide with
the sum $\measuredangle A O B+\measuredangle B O C$ up to full rotation.
This property will be expressed by the formula 
$$\measuredangle A O B+\measuredangle B O C\equiv \measuredangle A O C,$$
where \index{34@$\equiv$}``$\equiv$'' is a new notation which we are about to introduce.
The last identity will become a part of the axioms.

We will write $\alpha\equiv\beta\pmod{2\cdot\pi}$, or briefly
\begin{align*}
\alpha&\equiv\beta
\end{align*}
if $\alpha=\beta+2\cdot\pi\cdot n$
for some integer~$n$.
In this case we say 
$$\textit{``$\alpha$ is equal to $\beta$ modulo $2\cdot\pi$''}.$$
For example 
$$-\pi
\equiv
\pi\equiv 3\cdot\pi
\quad
\text{and}
\quad
\tfrac12\cdot\pi
\equiv
-\tfrac32\cdot\pi.$$

The introduced relation ``$\equiv$'' behaves as an equality sign,
but
\[\dots\equiv\alpha-2\cdot\pi\equiv \alpha\equiv \alpha+2\cdot\pi\equiv \alpha+4\cdot\pi\equiv\dots;\] 
that is, if the angle measures differ by full turn,
then they are considered to be the same.

With ``$\equiv$'', we can do addition, subtraction, and multiplication with integer numbers without getting into trouble.
That is, if
$$\alpha\equiv\beta
\quad
\text{and}
\quad
\alpha'\equiv \beta',$$ 
then
$$\alpha+\alpha'\equiv\beta+\beta',
\quad
\alpha-\alpha'\equiv \beta-\beta'
\quad 
\text{and}
\quad
n\cdot\alpha\equiv n\cdot\beta$$
for any integer~$n$.
But ``$\equiv$'' does not in general respect multiplication with non-integer numbers; for example 
$$\pi
\equiv 
-\pi
\quad
\text{but}
\quad
\tfrac12\cdot\pi
\not\equiv
-\tfrac12\cdot\pi.$$ 

\begin{thm}{Exercise}\label{ex:2a=0}
Show that $2\cdot\alpha\equiv0$ if and only if $\alpha\equiv0$ or $\alpha\equiv\pi$.
\end{thm}

\section*{Continuity}
\addtocontents{toc}{Continuity.}

The angle measure is also assumed to be continuous.
Namely, the following property of angle measure will become a part of the axioms:

\textit{The function}
$$\measuredangle\:(A,O,B)\mapsto\measuredangle A O B$$
\textit{is continuous at any triple of points $(A,O,B)$
such that $O\ne A$ and $O\ne B$ and $\measuredangle A O B\ne\pi$.}

To explain this property, we need to extend the notion of {}\emph{continuity} to the functions between metric spaces.
The definition is a straightforward generalization of the standard definition for the real-to-real functions.

Further, let $\mathcal X$ and $\mathcal Y$ be two metric spaces,
and $d_{\mathcal X}$, $d_{\mathcal Y}$ be their metrics.

A map $f\:\mathcal X\to\mathcal Y$ is called \index{continuous}\emph{continuous} at point $A\in \mathcal X$
if for any $\epsilon>0$ there is $\delta>0$, such that 
\[d_{\mathcal X}(A,A')
<
\delta
\quad
\Rightarrow
\quad
d_{\mathcal Y}(f(A),f(A'))
<
\epsilon.\]
(Informally it means that sufficiently small changes of $A$ result in arbitrarily small changes of $f(A)$.)

A map $f\:\mathcal X\to\mathcal Y$ is called \index{continuous}\emph{continuous} if it is continuous at every point $A\in \mathcal X$.

One may define a continuous map of several variables the same way.
Assume $f(A,B,C)$ is a function which returns a point in the space $\mathcal Y$ for a triple of points $(A,B,C)$
in the space~$\mathcal X$.
The map $f$ might be defined only for some triples in~$\mathcal X$.

Assume $f(A,B,C)$ is defined.
Then, we say that $f$ is continuous at the triple $(A,B,C)$ 
if for any $\epsilon>0$ there is $\delta>0$ such that 
\[d_{\mathcal Y}(f(A,B,C),f(A',B',C'))<\epsilon.\]
if $d_{\mathcal X}(A,A')<\delta$, $d_{\mathcal X}(B,B')<\delta$, and $d_{\mathcal X}(C,C')<\delta$.


\begin{thm}{Exercise}\label{ex:dist-cont}
Let $\mathcal{X}$ be a metric space.
\begin{enumerate}[(a)]
\item\label{ex:dist-cont:a} Let $A\in \mathcal{X}$ be a fixed point.
Show that the function 
$$f(B)\df
d_{\mathcal{X}}(A,B)$$ 
is continuous at any point~$B$.
\item Show that $d_{\mathcal{X}}(A,B)$ is continuous at any pair $A,B\in \mathcal{X}$.
\end{enumerate}

\end{thm}

\begin{thm}{Exercise}\label{ex:comp+cont}
Let $\mathcal{X}$, $\mathcal{Y}$, and $\mathcal{Z}$ be a metric spaces.
Assume that the functions $f\:\mathcal{X}\to\mathcal{Y}$
and $g\:\mathcal{Y}\to\mathcal{Z}$ are continuous at any point,
and $h=g\circ f$ is their composition;
that is, $h(A)=g(f(A))$ for any $A\in \mathcal{X}$.
Show that $h\:\mathcal{X}\to\mathcal{Z}$ is continuous at any point.
\end{thm}

\begin{thm}{Exercise}\label{ex:isom-cont}
Show that any distance-preserving map is continuous at any point.
\end{thm}




\section*{Congruent triangles} 
\addtocontents{toc}{Congruent triangles.}

Our next goal is to give a rigorous meaning for (\ref{preaxiomIV}) on page \pageref{preaxiomIV}.
To do this, we introduce the notion of {}\emph{congruent triangles}
so instead of {}\emph{``if we rotate or shift we will not see the difference''} we say that for triangles, the side-angle-side congruence holds;
that is, two triangles are congruent if they have two pairs of equal sides and the same angle measure between these sides.

An {}\emph{ordered} triple of distinct points in a metric space $\mathcal{X}$, 
say $A,B,C$,
is called a \index{triangle}\emph{triangle $ABC$}\label{page:def:triangle} (briefly \index{20@$\triangle$}$\triangle A B C$).
Note that the triangles $A B C$ and $A C B$ are considered as different.

Two triangles $A' B' C'$ and $A B C$ are called 
\index{triangle!congruent triangles}
\index{congruent triangles}\emph{congruent}
(it can be written as \index{32@$\cong$}$\triangle A' B' C'\z\cong\triangle A B C$) if there is a motion $f\:\mathcal{X}\to\mathcal{X}$ such that 
\[A'\z=f(A),
\quad
B'=f(B)
\quad
\text{and}
\quad
C'=f(C).\]

Let $\mathcal X$ be a metric space,
and $f,g\:\mathcal X\to\mathcal X$ be two motions.
Note that the inverse $f^{-1}:\mathcal X\to\mathcal X$,
as well as the composition $f\circ g:\mathcal X\to\mathcal X$
are also motions.

It follows that ``$\cong$'' is an \index{equivalence relation}\emph{equivalence relation};
that is, any triangle congruent to itself, 
and the following two conditions hold:
\begin{itemize} 
\item If $\triangle A' B' C'\z\cong\triangle A B C$, then $\triangle A B C\z\cong\triangle A' B' C'$.
\item If $\triangle A'' B'' C''\z\cong\triangle A' B' C'$ and $\triangle A' B' C'\z\cong\triangle A B C$,
then 
$$\triangle A'' B'' C''\cong\triangle A B C.$$
\end{itemize}


Note that if $\triangle A' B' C'\z\cong\triangle A B C$,
then $AB\z=A'B'$,
$BC=B'C'$ and $CA=C'A'$.

For a discrete metric, as well as some other metrics, 
the converse also holds.
The following example shows that it does not hold in the Manhattan plane:

\parbf{Example.}\label{example:isometric but not congruent} Consider three points 
$A=(0,1)$, $B=(1,0)$, and $C\z=(-1,0)$ on the Manhattan plane $(\mathbb{R}^2,d_1)$.
Note that
$$d_1(A,B)=d_1(A,C)=d_1(B,C)=2.$$

On one hand,
$$\triangle ABC\cong \triangle ACB.$$
Indeed, the map $(x,y)\z\mapsto (-x,y)$ is a motion of $(\mathbb{R}^2,d_1)$
that sends $A\z\mapsto A$, $B\mapsto C$, and $C\z\mapsto B$.

\begin{wrapfigure}[6]{o}{33mm}
\vskip-4mm
\centering
\includegraphics{mppics/pic-6}
\end{wrapfigure}

On the other hand,
$$\triangle ABC\ncong \triangle BCA.$$
Indeed, arguing by contradiction, assume that $\triangle ABC\cong \triangle BCA$; that is, there is a motion $f$ of $(\mathbb{R}^2,d_1)$ that sends $A\mapsto B$, $B\mapsto C$, and $C\mapsto A$.


We say that $M$ is a midpoint of $A$ and $B$ if 
\[d_1(A,M)=d_1(B,M)=\tfrac12\cdot d_1(A,B).\]
Note that a point $M$ is a midpoint of $A$ and $B$ if and only if $f(M)$ is a midpoint of $B$ and~$C$.

The set of midpoints for $A$ and $B$ is infinite, it contains all points $(t,t)$ for $t\in[0,1]$ (it is the gray segment on the picture above).
On the other hand, the midpoint for $B$ and $C$ is unique (it is the black point on the picture).
Thus, the map $f$ cannot be bijective --- a contradiction.

\addtocontents{toc}{\protect\end{quote}}
