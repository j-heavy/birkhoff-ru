\chapter*{Introduction}
\addcontentsline{toc}{chapter}{Introduction}
\addtocontents{toc}{\protect\begin{quote}}

Эта книга должна быть
строгой, 
консервативной, 
элементраной и
минималистичной.
В то же время, она включает в себя примерно максимум того, что студенты могут усвоить за один семестр.

Примерно одна треть материала изучалась в старшей школе, но теперь это не так.

Настоящая книга основана 
на курсе, прочитанном автором
в университете Пенсильвании
как введение в основания геометрии.
Лекции были ориентированы на второкурсников и студентов старших курсов университета.
У этих студентов уже был курс математического анализа.
В частности, они знакомы с действительными числами и понятием непрерывности.
Это позволяет проходить материал гораздо быстрее и строже, нежели в классах старшей школы.

\section*{Необходимые знания}
\addtocontents{toc}{Prerequisite.}

Студент должен быть знаком со следующими разделами математики:
\begin{itemize}
\item Наивная (Канторова) теория множеств: 
$\in$,
$\cup$, 
$\cap$,
$\backslash$,
$\subset$,~$\times$.
\item Действительные числа: интервалы, неравенства, алгебраические тождества.
\item Предел, непрерывные функции и теорема о промежуточном значении (Больцано — Коши).
\item Стандартные функции: 
абсооютная величина (модуль), 
натуральный логарифм,
экспоненциальная функция. 
Иногда будут встречать тригонометрические функции, 
но эти части можно пропустить.
\item  В главе~\ref{chap:trans} встречаются основы линейной алгебры.
\item Для~\ref{chap:sphere}, будет плюсом умение работать со {}\emph{скалярным произвидением}.
\item Для~\ref{chap:complex}, также не будет лишним понимание комплексных чисел.
\end{itemize} 

\section*{Структура книги}
\addtocontents{toc}{Overview.}

Мы будем использовать {}\emph{метрический подход} предложенный Биркгофым.
Это означает, что мы определяем Евклидово пространство как {}\emph{метрическое}, которое имеет определённые свойства ({}\emph{аксиомы}).
Таким образом, мы минимизируем утомительные части,
которые являются неизбежными в подходе Гильберта.
В то же время, студенты имеют возможность изучить основы метрических пространств.

Иллюстрация структуры:

\begin{figure}[h!]
\centering
\begin{tikzpicture}[->,>=stealth',shorten >=1pt,auto,scale=1.4,
  thick,main node/.style={circle,draw,font=\sffamily\bfseries,minimum size=8mm}]

  \node[main node] (1) at (1,15/6) {\ref{chap:metr}};
  \node[main node] (2) at (2,15/6){\ref{chap:axioms}};
  \node[main node] (3) at (3,15/6) {\ref{chap:half-planes}};
  \node[main node] (4) at (4,15/6) {\ref{chap:cong}};
  \node[main node] (5) at (3.5,10/6) {\ref{chap:perp}};
  \node[main node] (6) at (4.5,10/6) {\ref{chap:parallel}};
  \node[main node] (61) at (4,5/6) {\ref{chap:angle-sum}};
  \node[main node] (7) at (5,5/6) {\ref{chap:triangle}};
  \node[main node] (8) at (3,5/6){\ref{chap:inscribed-angle}};
  \node[main node] (9) at (2,5/6) {\ref{chap:inversion}};
  \node[main node] (10) at (2.5,10/6) {\ref{chap:non-euclid}};
  \node[main node] (11) at (1.5,10/6){\ref{chap:poincare}};
  \node[main node] (12) at (.5,10/6) {\ref{chap:h-plane}};
  \node[main node] (13) at (1.5,0) {\ref{chap:trans}};
  \node[main node] (14) at (0.5,0) {\ref{chap:proj}};
  \node[main node] (15) at (1,5/6) {\ref{chap:sphere}};
  \node[main node] (16) at (0,5/6) {\ref{chap:klein}};
  \node[main node] (17) at (2.5,0) {\ref{chap:complex}};
  \node[main node] (18) at (3.5,0) {\ref{chap:car}};
  \node[main node] (19) at (4.5,0) {\ref{chap:area}};

  \path[every node/.style={font=\sffamily\small}]
   (1) edge node[right]{}(2)
   (2) edge node[right]{}(3)
   (3) edge node[right]{}(4)
   (4) edge node[right]{}(5)
   (5) edge node[right]{}(6)
   (6) edge node[right]{}(61)
   (61) edge node[right]{}(8)
   (61) edge node[right]{}(7)
   (61) edge node[right]{}(19)
   (9) edge node[right]{}(13)
   (13) edge node[right]{}(14)
   (61) edge node[right]{}(18)
   (8) edge node[right]{}(9)
   (9) edge node[right]{}(11)
   (9) edge node[right]{}(15)
   (9) edge node[right]{}(17)
   (5) edge node[right]{}(10)
   (10) edge node[right]{}(11)
   (11) edge node[right]{}(12)
   (14) edge node[right]{}(16)
   (15) edge node[right]{}(16)
   (12) edge node[dashed,right]{}(16);
\end{tikzpicture}
\end{figure}


В (\ref{chap:metr}) мы даем все определения, необходимые для формулирования аксиом:
понятие метрического пространства, 
линии, 
угла, 
непрерывного отоброжения и конгруэтных треугольников.


Далее мы переходим к Евклидовой геометрии:
(\ref{chap:axioms}) Аксиомы и следствия из них;
(\ref{chap:half-planes}) Полуинтервалы и непрерывность;
(\ref{chap:cong}) Конгруэнтные треугольники;
(\ref{chap:perp}) Окружности, движение, перпендикулярные линии;
(\ref{chap:parallel}) Подобные треугольники и (\ref{chap:angle-sum}) Параллельные прямые  
--- это первые две главы, где мы используем аксиомы ~\ref{def:birkhoff-axioms:4}, эквивалентные пятому постулату Евклида.
В (\ref{chap:triangle}) мы даём самые классические теоремы о треугольниказ;
эта глава включена, главным образом, в качестве иллюстрации.


В следующих двух главах мы обсудим геометрию окружностей:
(\ref{chap:inscribed-angle}) Вписанные углы; (\ref{chap:inversion}) Инверсии.
Это будет использовано для построения модели гиперболической геометрии.

В дальнейшем, 
мы обсудим не-Евклидову геометрию:
(\ref{chap:non-euclid})
Нейтральня геометрия --- геометрия без аксиомы параллельности;
(\ref{chap:poincare})
Модель конформного диска ---
это построение гиперболической геометрии,
пример нейтральной плоскости, не являющейся Евклидовой.
В (\ref{chap:h-plane}) мы обсудим гиперболическую --- конечная точка книги.

В оставшихся главах мы обсудим дополнительные разделы:
(\ref{chap:trans}) Аффинная геометрия;
(\ref{chap:proj}) Проективная геометрия;
(\ref{chap:sphere}) Сферическая геометрия;
(\ref{chap:klein}) Проективная модель гиперболического пространства;
(\ref{chap:complex}) Комплексные координаты;
(\ref{chap:car}) Геометрические конструкции;
(\ref{chap:area}) Площадь.
Доказательства в этих главах недостаточны строги.

Мы рекомендуем использовать визуальные задания на сайте автора.

\section*{Дисклеймер}

Невозможно найти оригинальную ссылку на большинство обсуждаемых здесь теорем, поэтому я даже не пытаюсь это сделать.
Большинство доказательств, обсуждаемых в книге,
уже были представлены в Началах Евклида.

\section*{Рекомендуемая литература}

\begin{itemize}
\item Byrne's Euclid \cite{byrne} --- a coloured version of the first six books of Euclid's Elements edited by Oliver Byrne. 

\item Kiselev's textbook \cite{kiselev} ---
a classical book for school students; it should help if you have trouble following this book.

\item Hadamard's book \cite{hadamard} --- an encyclopedia of elementary geometry originally written for school teachers.

%\item Moise's book, \cite{moise} --- should be good for further study.

%\item Greenberg's book \cite{greenberg}  --- a historical tour in the axiomatic systems of various geometries.

\item Prasolov's book \cite{prasolov} is perfect to master your problem-solving skills.

\item Akopyan's book \cite{akopyan} --- a collection of problems formulated in figures.

\item Methodologically my lectures
were very close to Sharygin's  textbook \cite{sharygin}.
This is the greatest textbook in geometry for school students,
I recommend it to anyone who can read Russian.


\end{itemize}

\section*{Acknowlegments}

Let me thank  
Matthew Chao, 
Svetlana Katok, 
Alexander Lytchak,
Alexei Novi\-kov,
and Lukeria Petrunina
for useful suggestions and correcting the misprints.






\addtocontents{toc}{\protect\end{quote}}
